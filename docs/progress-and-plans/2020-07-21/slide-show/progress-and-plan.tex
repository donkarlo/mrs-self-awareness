% !TeX TXS-program:bibliography = txs:///biber
\documentclass[unknownkeysallowed]{beamer}
\usetheme{UniKlu}
\usepackage[backend=biber,style=apa,sorting=nty, bibencoding=utf8]{biblatex}
\addbibresource{/home/donkarlo/Dropbox/projs/research/refs.bib}


\usepackage{xcolor}

\title{Collective Self-awareness in Progress and Plans}
\author{Mohammad Rahmani}
\institute{DECIDE Doctoral School}

\begin{document}
	\begin{frame}
		\maketitle
	\end{frame}

	\begin{frame}{Dynamic systems}
		\begin{figure}
			\includegraphics[scale=1]{/home/donkarlo/Dropbox/projs/research/assets/generalized-state-space.jpg}
			\caption{Generalized space state}
		\end{figure}
	\end{frame}

	\begin{frame}{Semantic SA - State space segmentation}
		To either capture traits such as left turning of a moving object along time courses or constant motion zones 
		\begin{itemize}
			\item Self-organizing maps (SOM)
			\item Growing Neural Gas (GNG)
			\item Gaussian process division (GP)
		\end{itemize}
	\end{frame}

	\begin{frame}{Generative/predictive models}
		Filtering (State estimation from observation):
		\begin{itemize}
			\item \textbf{Kallman Filter (KF):} Continuous state prediction for linear, dynamic systems from incremental observation.
			\item \textbf{Particle Filter (PF) - Sequential Monte Carlo (SMC):} Discrete state prediction for non-linear, none Gaussian (error-wise) dynamic systems
		\end{itemize}
		Switching models (Next state estimation according to current observation):
		\begin{itemize}
			\item \textbf{Markov Jump Linear Systems (MJLS):} For linear dynamic systems with continuous state space
			\item \textbf{Markov Jump Particle Filter (MJPF):} For dynamic, none linear system with any noise distribution which contains:
			\begin{itemize}
				\item Bank of KFs to map continuous observation to states
				\item Banks of PFs to map discrete observations to states
			\end{itemize}
		\end{itemize}
	\end{frame}

	\begin{frame}{Generative/predictive models}
		A very rough sketch of existing models:
		\begin{figure}
			\includegraphics[scale=0.3]{/home/donkarlo/Dropbox/projs/research/assets/transition-from-observation-to-words.png}
			\caption{}
		\end{figure}
	\end{frame}

	\begin{frame}{Semantic SA - Words as synchronously happened static motion zones}
		Could be used in anomaly detection of a force field driven environment
		\begin{itemize}
			\item \textbf{Semantic segmentation}: SOM
			\item \textbf{Alphabet level}: Zones/prototypes where the motion remains constant
			\item \textbf{Word level}: Simultaneous occurrence of alphabets in a reference(training) task
			\item \textbf{Sentence/Grammar level}: The sequence of words from the beginning of an interaction to the end
		\end{itemize}
		\begin{figure}
			\includegraphics[scale=0.7]{/home/donkarlo/Dropbox/projs/research/assets/mutial-experienced-semantics.jpg}
			\caption{}
		\end{figure}
	\end{frame}

	\begin{frame}{Semantic SA - Words as classes of generalized state traits}
		\begin{itemize}
			\item \textbf{Semantic segmentation}: Self-organizing maps (SOM) / Growing Neural Gas (GNG)
			\item \textbf{Alphabet level}: Centroids of time's i-th derivative of states derived from resulting clusters in state space segmentation phase
		\end{itemize}
		\begin{figure}
			\includegraphics[scale=0.7]{/home/donkarlo/Dropbox/projs/research/assets/trait-based-alphabets.jpg}
			\caption{}
		\end{figure}
	\end{frame}

	\begin{frame}{Semantic SA - Words as classes of generalized state traits - 2}
		\begin{itemize}
			\item \textbf{Word level}: A set of alphabets containing all time derivative orders 
			\begin{equation}
			w = \{\alpha^{(0)},...,\alpha^{(L)}\}
			\end{equation}
			where $\alpha^i$ is the centroid of i-th time derivative order of the states in a cluster.
			\item \textbf{Sentence/Grammar level}: The sequence of words formed in a reference task
		\end{itemize}
	\end{frame}

	\begin{frame}{Future plans for Words as classes of generalized state traits}
		Building words/semantics on sensor heterogeneity level
		\begin{itemize}
			\item Different sensors (e.g one agent Temperature)
		\end{itemize}
	\end{frame}

	\begin{frame}{Future plan: Investigating applications in the following areas:}
		\begin{itemize}
			\item Internet of things (IoT)
			\item Cyber physical systems (CPS)
			\item Swarm intelligence
		\end{itemize}
	\end{frame}
	\begin{frame}{Future plan for synchronously activated motion zones}
		\begin{itemize}
			\item Relating Reinforcement learning(RL) with Force-field(FF) analysis
				\begin{itemize}
					\item Rewards will be modeled by attractive forces
					\item Costs will be modeled by repulsive forces
				\end{itemize}
		\end{itemize}
		\textbf{Question?} Can semantic state description from robots help them to adapt gradually to new versions of the task they are trying to accomplish by better cooperation using RL and FF? 
	\end{frame}

	\begin{frame}{Future plan - words as traits - 1}
		\begin{itemize}
			\item Similar word in different sequences must induce new actions (Semantic context awareness)
		\end{itemize}
		Such as in driving
		\begin{itemize}
			\item Taking over
				\begin{itemize}
					\item Attractor vehicle  = Turning left + increase speed + Turning right
					\item Follower vehicle = nothing + decrease speed a little + nothing
				\end{itemize}
		\end{itemize}
		\begin{figure}
			\includegraphics[scale=0.5]{/home/donkarlo/Dropbox/projs/research/assets/word-traits-cars-three.jpg}
			\caption{Model creation from anomaly}
		\end{figure}
	\end{frame}

	\begin{frame}{Future plans - words as traits - 2}
		\begin{itemize}
			\item Getting out of park:
			\begin{itemize}
				\item Attractor vehicle = Turn left + increase speed
				\item Follower vehicle =  move forward a little
			\end{itemize}
		\end{itemize}
		\begin{figure}
			\includegraphics[scale=0.5]{/home/donkarlo/Dropbox/projs/research/assets/word-traits-cars-two.jpg}
			\caption{}
		\end{figure}
		Turning left in both examples are the same but they must partially contribute to inducing different actions. 
	\end{frame}


	\begin{frame}{Other future plans to investigate - 1}
		\begin{itemize}
			\item Unifying "Words as traits" and "synchronously activated motion zones" into one idea to build more efficient CA system.
			\item What other strategies can be taken to suggest words which improve semantic awareness?
		\end{itemize}
	\end{frame}
	\begin{frame}{Other future plans to investigate - 2}
		\begin{itemize}
			\item How to relate hierarchical composition of semantics from one agent to actions in another agent?
			\item How to prove semantic awareness of IAs improves the balance in goals of a system?
		\end{itemize}
	\end{frame}

	\begin{frame}{Other future plans to investigate - 3}
		\begin{itemize}
			\item How does semantic-awareness help with improvements in minimum AI SA requirements such as initialization, memorization, predictive model temporal-causal model creation, anomaly detection and decision making?  
		\end{itemize}
	\end{frame}
\end{document}
