\documentclass{article}
\usepackage{comment}
\usepackage[english]{babel}
\usepackage[utf8]{inputenc}
\usepackage{fancyhdr}
\usepackage[round]{natbib}
\usepackage{graphicx}
\usepackage{url}
\usepackage{amsmath}
\usepackage{amssymb}
\DeclareMathOperator*{\argmax}{argmax}
\DeclareMathOperator*{\argmin}{argmin}
\pagenumbering{arabic}
\usepackage{multicol}
\usepackage{siunitx}
\usepackage{soul}

\pagestyle{fancy}
\fancyhf{}
\rhead{Mohammad Rahmani}
\lhead{From perception to CA}

\newcommand{\ignore}[1]{}
\begin{document}
	\bibliographystyle{plainnat}
	\title{From individual perception to collective behavior in drones. A self-aware approach}
	\author{Mohammad Rahmani}
	\date{}
	\maketitle
	
	\paragraph{} An approach in self-awareness of intelligent agents, implies them to develop a capability in making a distinction between their initial knowledge and new experiences such that they can store and retrieve these experiences in form of predictive models in order to maintain their homeostasis conditions. Such capability embodies in the ability to detect anomalies in sensory data in comparison to the data predicted by these models. 
	
	\paragraph{}In a collection of agents, emergence of such anomaly, not only is needed to be detected by individual agents from environmental data, but also from the motion of other agents in the collection. One such kind of decisions may contribute to taking a new collective formation to overcome the new circumstances such as obstacles. Otherwise the homeostasis level of the collection as whole, reduces. 
	
	Due to the high computational complexity of anomaly detection from continuous data, novel methods should be sought to improve such efficiency. These solutions are mostly inspired by collective behavior of biological entities in nature such as ants and bees. Since such biological entities do not communicate by composing meaningful pieces to build new meanings, this area is left widely unexplored.
	
	\paragraph{} Imagine a set of multi-rotor drones trained to transport a cable-suspended rigid payload from one point to another on a straight line. If this initial knowledge is interrupted by a vertical column on the left side of the trajectory, then an agent's collision avoidance system enforces a left shift to avoid collision. This state change can be described to neighboring drones who could consequently react and inform others until all agents shift to left. As such lining in a row to pass through a narrow corridor or taking distance from each other to reduce the height of the system including the payload to pass through a short window, etc could be inferred out of such local communications.
	
	\paragraph{} As an attempt to address the aforementioned necessity, this abstract suggest segmenting generalized state space of the initial knowledge that each agent should accomplish into meaningful segments such that composition of those segments describe new but meaningful states. The same approach could be taken toward describing similar states for a particular collective formation along the time. The emergence of local communications in the former language may raise a sign for occurrence of abnormality in the current formation and a sign for the next appropriate formation described by the second language. 
	
	\paragraph{} An approach in self-awareness which is based on building and storing new predictive, dynamic state models from a new behavior of a collection of agents by detecting anomalies between current observations of the agents and predictions of similar, but previously stored models. 
	
	
\end{document}