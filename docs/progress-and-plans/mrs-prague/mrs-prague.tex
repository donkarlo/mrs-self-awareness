% !TeX document-id = {412070ce-2b89-4575-900e-e9d3183d0ba4}
% !TeX TXS-program:bibliography = txs:///biber
\documentclass[unknownkeysallowed]{beamer}
\usetheme{UniKlu}
\usepackage[backend=biber,style=apa,sorting=nty, bibencoding=utf8]{biblatex}
\addbibresource{/home/donkarlo/Dropbox/projs/research/refs.bib}


\usepackage{xcolor}

\title{From Individual Perception to Collective Behavior in UAVs. A self-aware approach}
\author{Mohammad Rahmani}
\institute{Pervasive Computing Group (Klagenfurt university)}

\begin{document}
	\begin{frame}
		\date{}
		\maketitle
		\textcolor{white}{\textbf{17 September 2020}}
	\end{frame}

	\begin{frame}{Content}
		\begin{itemize}
			\item Concepts and methods
			\begin{itemize}
				\item A Dynamic Bayesian Network (DBN) approach in Self-awareness (SA) and characteristics of a self-aware Intelligent Agent (IA) 
			\end{itemize}
			\item A proposal for Collective Self-awareness (CA) application in a multi-UAV system
		\end{itemize}
	\end{frame}
	
	\begin{frame}{Intelligent Agents (IA), Sensors and Actuators}
		Each IA, either biological or artificial, incorporates:
		\begin{itemize}
			\item Sensors
				\begin{itemize}
					\item Proprioceptive (Cochlea, IMU)
					\item Exteroceptive (Eyes, Camera)
				\end{itemize}
			\item Actuators (Feet, Engine)
		\end{itemize}
	\end{frame}

	\begin{frame}{Self-awareness (SA)}
		SA is an approach in Artificial Intelligence to enable IAs to make a distinction between their previous experiences and new experiences observed by the sensors (Abnormality detection) \footnote{\fullcite{regazzoni-2020-multi-sensorial-generative-and-descriptive-self-awareness-models-for-autonomous-systems}} and
		\begin{itemize}
			\item build predictive models from these new experiences 
			\item store them and retrieve them to predict and plan future
			\item Make appropriate decisions such as disturbance rejection or path re-planning
		\end{itemize}
		
	\end{frame}

	\begin{frame}{Simple Illustration of an SA drone}
		\begin{figure}
			\includegraphics[scale=0.06]{/home/donkarlo/Dropbox/projs/research/assets/drone-sa.png}
			\caption{SA, sensors and actuators}
		\end{figure}
	\end{frame}

	\begin{frame}{The Ultimate Goal of a Self-aware IA}
		\begin{itemize}
			\item To maintain its homeostasis condition over the course of time by taking advantage of the modeled experiences to improve issues such as 
			\begin{itemize}
				\item Resource management
				\item Security
				\item Safety
			\end{itemize}
		\end{itemize}
	\end{frame}

	\begin{frame}{Collective Self-awareness (SA)}
		\begin{itemize}
			\item The ability to detect abnormality in the course of relation a couple of IAs were supposed to maintain and make appropriate decisions to improve collective homeostasis conditions
				\begin{itemize}
					\item \textbf{Example}: Taking an appropriate \textbf{formation} when the collection faces a factor detrimental to its collective homeostasis condition
				\end{itemize}
		\end{itemize}
	\end{frame}

	\begin{frame}{Tools and Theories to Implement SA/CA}
		\begin{itemize}
			\item Dynamic Bayesian Networks such as 
			\begin{itemize}
				\item Markov Jump Linear Systems (KF+DBN) \footnote{\url{doucet-2001-particle-filters-for-state-estimation-of-jump-markov-linear-systems}}
				\item Markov Jump Particle (KF+PF+DBN) Systems\footnote{\url{baydoun-2018-learning-switching-models-for-abnormality-detection-for-autonomous-driving}}
			\end{itemize}
			\item Force Field Analysis (Autonomous navigation)
			\item Variational Auto Encoders (To generate different versions of the same experience)
			\item Continual/Lifelong learning (To include all experiences in one model)
		\end{itemize}
	\end{frame}

	\begin{frame}{SA in Single UAV Navigation for Aerial Manipulation}
		The aforementioned abilities in a single UAV navigation in tight spaces such as buildings translates to:
		\begin{itemize}
			\item Path/motion planning
			\item State estimation
			\item Trajectory tracking
				\begin{itemize}
					\item \textbf{Minor anomaly detection} Disturbance rejection
				\end{itemize}
			\item \textbf{Major anomaly detection}: Collision avoidance
				\begin{itemize}
					\item Corridor turning points
					\item Vertical collision avoidance
					\item Horizontal collision avoidance
				\end{itemize}
		\end{itemize}
	\end{frame}

	\begin{frame}{CA in Multi UAV Navigation for Aerial Manipulation}
		Not only each individual IA must be SA, but also the whole collection should include these abilities:
		\begin{itemize}
			\item \textbf{Collective path/motion planning}
			\item \textbf{Formation state estimation}
			\item \textbf{Formation anomaly detection}: While individuals perform collision avoidance maneuvers and taking the right decision toward a new appropriate formation to avoid load and system collision
		\end{itemize}
	\end{frame}

	\begin{frame}{CA scenarios}
		CA formation models from which appropriate actions should be practiced
		\begin{figure}
			\includegraphics[scale=0.05]{/home/donkarlo/Dropbox/projs/research/assets/colletive-all-scenarios.png}
			\caption{}
		\end{figure}
	\end{frame}

	\begin{frame}{First Language: Individual Semantic Emergence}
		Discretized\footnote{\fullcite{fiser-2013-growing-neural-gas-efficiently}} generalized state for different derivatives of time, forms the alphabet of words by which each individual agent can describe the experiences it is practicing to other agents \footnote{\fullcite{kanapram-2019-self-awareness-in-intelligent-vehicles-experience-based-abnormality-detection}}
		\begin{figure}
			\includegraphics[scale=0.4]{/home/donkarlo/Dropbox/projs/research/assets/trait-based-alphabets.jpg}
			\caption{}
		\end{figure}
		\begin{equation}
			w = \{\alpha^{(0)},...,\alpha^{(L)}\}
		\end{equation}
	\end{frame}

	\begin{frame}{Second language: Collective Semantic Emergence}
		Mutually activated discretized generalized state space form the collective language which can describe the relation(Formation) a collection of agents are supposed to maintain over the course of time \footnote{\fullcite{baydoun-2019-prediction-of-multi-target-dynamics-using-discrete-descriptors-an-interactive-approach}}
		\begin{figure}
			\includegraphics[scale=0.7]{/home/donkarlo/Dropbox/projs/research/assets/mutial-experienced-semantics.jpg}
			\caption{}
		\end{figure}
	\end{frame}

	\begin{frame}{Future plan}
		How should emergence and frequency of locally communicated phrases of individual agent experiences in the first language be either 
		\begin{itemize}
			\item mapped to those of the second language to form predictive models
			\item or used as a new anomaly detection method upon which new predictive models can be extracted
		\end{itemize}
	\end{frame}

	\begin{frame}{Drone hub, Klagenfurt University UAV lab}
		\begin{itemize}
			\item {\color{blue} uav.aau.at}
			\item The largest indoor research space for aerial vehicles with the biggest tracking volume in Europe
		\end{itemize}
	\end{frame}
\end{document}
