% !TeX document-id = {412070ce-2b89-4575-900e-e9d3183d0ba4}
% !TeX TXS-program:bibliography = txs:///biber
\documentclass[unknownkeysallowed]{beamer}
\usetheme{UniKlu}
\usepackage[backend=biber,style=apa,sorting=nty, bibencoding=utf8]{biblatex}
\addbibresource{/home/donkarlo/Dropbox/projs/research/refs.bib}


\usepackage{xcolor}

\title{Progress and Plans}
\author{Mohammad Rahmani}
\institute{Decide Doctoral School}

\begin{document}
	\begin{frame}
		\date{}
		\maketitle
		\textcolor{white}{\textbf{19 October 2020}}
	\end{frame}
	
	\begin{frame}{Table of contents}
		\begin{itemize}
			\item Literature review on existing approaches in individual SA and CA
			\item Literature review on discretized generalized state space models
			\item Literature review on collective aerial transportation
			\item Literature review on formation control and consensus seeking
			\item Determining experimental scenarios
			\item Outlining probable questions to which this research must answer
			\item Attending Prague summer school
			\item Determining simulation environment
		\end{itemize}
	\end{frame}

	\begin{frame}{Literature review on existing approaches in SA and CA}
		Resulted in narrowing the research domain to
		\begin{itemize}
			\item adopting a definition of SA and CA which is based on modeling detected anomaly in dynamic systems as a predictive model
			\item training predictive DBN models from detected anomalies 
			\item methods for storing the models
			\item methods for retrieving the models
			\item combining the two previous steps using continual learning techniques
		\end{itemize}
	\end{frame}

	\begin{frame}{Literature review on discretized generalized state space models}
	 Resulted in determining two abstract communication methods to
		\begin{enumerate}
			\item describe generalized IA motions
			\item describe the expected relationship over the course of time which should be maintained between a collection of IAs 
		\end{enumerate}
	\end{frame}

	\begin{frame}{Literature review on collective aerial transportation}
		Resulted in determining two set of scenarios which
		\begin{itemize}
			\item define CA as modeling the detected anomaly in a previously agreed consensus between the agents
			\item avoid IAs dispersion around obstacles
			\item taking a dynamic leader-follower approach (similar to biological agents when transporting payloads)
		\end{itemize}
	\end{frame}

	\begin{frame}{Scenarios - 1}
		\textbf{Overview}: A collection of IAs that carry an imaginary payload with some degree of freedom from a starting point to a destination (attractive force) while some repulsive forces obstruct them in two forms
		\begin{itemize}
			\item Carry a payload between them
			\item Carry a payload which is suspended from a cable
		\end{itemize}
		Repulsive forces to overcome
		\begin{itemize}
			\item Vertical column
			\item Horizontal obstacles
			\item Changes in the corridor width and height (suspended form only)
			\item Turning points
		\end{itemize}
	\end{frame}

	\begin{frame}{Scenarios - 2}
		Determining the logic of the scenarios
		\begin{itemize}
			\item Each agent tries to individually avoid repulsive forces 
			\item Each agent describes the changes in its discreet motion space when trying to avoid repulsive forces to neighboring agents using the first communication's semantic segments
			\item such locally transmitted messages will be either mapped to a collective behavior described by a sequence of semantic segments of the second communication method or is detected as an anomaly in the collective behavior from which a new DBN model should be generated
		\end{itemize}
	\end{frame}

	\begin{frame}{Probable questions the scenarios should answer?}
		\begin{itemize}
			\item Does locally conversed motion changes detect anomalies better than state-of-the-art? Better means:
			\begin{itemize}
				\item earlier anomaly detection from existing collective behavior models
				\item generating larger anomaly signals to better differentiate them against disturbances such as sensor noises
			\end{itemize}
			\item Does modeling CA in the form of second abstraction language improves state-of-the-art in   homeostasis in swarms?
			\item Can a coupled encoder-decoder architecture improve state-of-the-art in improving catastrophic forgetting in ANNs? 
		\end{itemize}
	\end{frame}

	\begin{frame}{Prague summer school}
		Resulted in
		\begin{itemize}
			\item Presenting the aforementioned scenarios
			\item Using their Multi-UAV framework for simulating the scenarios
		\end{itemize}
	\end{frame}

	\begin{frame}{Simulation Environment}
		\begin{itemize}
			\item Simulink
			\item Matlab
			\item ROS + Airsim
		\end{itemize}
		MRS enabled frameworks
		\begin{itemize}
			\item RotorS
			\item Hector Quadrotors 
			\item ...
		\end{itemize}
		\textbf{ROS + GAZEBO} applied in a framework developed by CTU-MRS
	\end{frame}

	\begin{frame}{Questions}
		\begin{itemize}
			\item Questions?
		\end{itemize}
	\end{frame}
\end{document}
