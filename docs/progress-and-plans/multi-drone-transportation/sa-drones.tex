\documentclass{article}
\usepackage{comment}
\usepackage[english]{babel}
\usepackage[utf8]{inputenc}
\usepackage{fancyhdr}
\usepackage[round]{natbib}
\usepackage{graphicx}
\usepackage{url}
\usepackage{amsmath}
\usepackage{amssymb}
\DeclareMathOperator*{\argmax}{argmax}
\DeclareMathOperator*{\argmin}{argmin}
\pagenumbering{arabic}
\usepackage{multicol}
\usepackage{siunitx}
\usepackage{soul}

\pagestyle{fancy}
\fancyhf{}
\rhead{Mohammad Rahmani}
\lhead{SA drone swarm for transportation}

\newcommand{\ignore}[1]{}
\begin{document}
	\bibliographystyle{plainnat}
	\title{Self-aware drone swarm for transportation}
	\author{Mohammad Rahmani}
	\date{}
	\maketitle
	\section{Introduction}
	Collective load transportation
	
	\section{Related work}
	
	\section{Homeostatic}
	Keep maximum stability which means 
		\begin{itemize}
			\item keeping the most equal distance possible between the drones and the load
		\end{itemize}
		
	\section{Generative models}
		\subsection{Initial model - constant movement between two points}
		\subsection{Vertical frame maneuver}
		\subsection{Horizontal frame maneuver}
		\subsection{Small square inference challenge} can it solve the small square problem?
		
		\subsection{Individual models}
	\section{Discriminative models / Abnormality detection }
		\begin{itemize}
			\item 
		\end{itemize}
	
	\section{Sensors}
		\subsection{Exteroceptive}
		\subsection{Proprioceptive}
		\subsection{Heterogeneity}
		\subsection{Multi-sensory approach / Fusion}
		\subsection{Feature selection for generative and discriminative models}
	
	\section{Leader-follower}
		\paragraph{Ranking}
		The one which is closer to the destination will be ranked lower and the ranking starts from 1. The leader is rank 1
	
	\section{Interaction}
		\section{Neighboring interaction}
			\begin{itemize}
				\item Only with the two neighbors on the sides
				\item in 3/2 the neigbouring ranks can do that.
			\end{itemize}
		\section{Semantic interaction}
		Understanding the meaning of an action by considering in the context of the actions it has happened and reacting according to that context. For example if a drone moves left, then should the neighbouring drones consider it as a local decision or as a decision to which they should react? 
		

	\section{Null hypothesis}
		\begin{itemize}
			\item Semantic interaction does not improve decision making
		\end{itemize}
	
	\section{Decision making and control}
		\begin{itemize}
			\item A model to convert abnormality signals to decisions for actuators 
		\end{itemize}
	
	\section{Roadmap}
	
	\bibliography{/home/donkarlo/Dropbox/projs/research/refs.bib}
\end{document}