% !TeX TXS-program:bibliography = txs:///biber
\documentclass[unknownkeysallowed]{beamer}
\usetheme{UniKlu}
\usepackage[backend=biber,style=apa,sorting=nty, bibencoding=utf8]{biblatex}
\addbibresource{/home/donkarlo/Dropbox/projs/research/refs.bib}


\usepackage{xcolor}

\title{Proposal experimental plan draft - Version 0.1}
\author{Mohammad Rahmani}
\institute{DECIDE Doctoral School}

\begin{document}
	\begin{frame}
		\date{}
		\maketitle
		\textcolor{white}{\textbf{12 August 2020}}
	\end{frame}

	\begin{frame}{Proposal Goal}
		\textbf{Scientific}
		\begin{itemize}
			\item Investigating whether intelligent agents' semantic-awareness help with emergence of better collective behavior, better means
			\begin{itemize}
				\item Maintaining higher levels of homeostasis
			\end{itemize}
		\end{itemize}
		
		\textbf{Technical}
		\begin{itemize}
			\item Scalable object transportation
		\end{itemize}
		\textbf{Inspired by}
		\begin{itemize}
			\item Swarm intelligence
		\end{itemize}
	\end{frame}

	\begin{frame}{Goal}
		\begin{itemize}
			\item \textbf{Primary}: Improvement in existing aerial cooperative load transportation
		\end{itemize}
	\end{frame}

	\begin{frame}{Semantic-awareness meaning}
		
		\begin{itemize}
			\item A mutual understanding of the meaning of different composition of the alphabet of the language which describes the generalized experienced/currently observed/future predicted (using generative models) (descritized) states. Example of the parts which could be composed together in a moving robot. For example: 
				\begin{itemize}
					\item Position
					\item Speed (first derivative of position change)
					\item Alteration (second derivative of position)
				\end{itemize} 
			\item Considering the meaning of different parts of this language in the context in which they appear to provide
				\begin{itemize}
					\item Temporality semantics
					\item Similarity semantics
				\end{itemize}
		\end{itemize}
	\end{frame}

	\begin{frame}{Goal - semantic perspective}
		Each agent receives sequences of generative models that neighboring agents have, are or will experience(d) and they decide for individual actions which emerges in a collective behavior to solve a problem.
	\end{frame}
	
	\begin{frame}{Related study area}
		\begin{itemize}
			\item Bayesian self-aware Artificial Intelligence for predictive model generation from huge abnormalities and decision making according to abnormality signals. 
			\item Collective adaptive systems
				\begin{itemize}
					\item Collective object transportation
				\end{itemize}
			\item Swarm navigation and Self-organization
			\item Dynamic system modeling
				\begin{itemize}
					\item Dynamic Bayesian modeling
				\end{itemize}
			\item Discretization of continuous features
			\item Semantics
		\end{itemize}
	\end{frame}

	\begin{frame}{Methodology}
		A Bayesian Artificial intelligence approach will be taken which must include
		\begin{itemize}
			\item Individual perception
			\item Discretization (using clustering methods) of state space and derive the alphabet of two languages
				\begin{itemize}
					\item Alphabets of words describing interaction between neighboring agents
					\item Alphabets of words describing individual dynamism 
				\end{itemize}
			\item Abnormality detection in individual motion
		\end{itemize}
	\end{frame}

	\begin{frame}{Methodology - 2}
		\begin{itemize}
			\item Abnormality detection in interaction
			\item Generative individual models
			\item Generative interaction models
			\item Descriminative models
			\item Control decision making according abnormality detection
		\end{itemize}
	\end{frame}

	\begin{frame}{Potential scenarios: Horizontal Frame Scenarios}
		Reference transportation
		\begin{figure}
			\centering
			\includegraphics[width=0.4\textwidth]{/home/donkarlo/Dropbox/projs/research/assets/swarm-drones-reference-task.png}
			\label{fig:swarm-drones-reference-task}
		\end{figure}
		The importance of such obstacles is that every surface can be divided to small, similar surfaces such that they approach the shape of the surface. 
	\end{frame}

	\begin{frame}{Potential scenarios: Horizontal Frame Scenarios - Frame passage transportation}
		New Generative DBN models can be learned out of the following scenarios. 
		\begin{figure}
			\centering
			\includegraphics[width=0.7\textwidth]{/home/donkarlo/Dropbox/projs/research/assets/vertical-frame-scenarios.png}
			\label{fig:collective-behaviour-learning}
		\end{figure}
	\end{frame}

	\begin{frame}{Potential scenarios: Horizontal Frame Scenarios}
		\begin{itemize}
			\item \textbf{Scenario 1}: Row formation to pass through a vertically narrow frame 
			\item \textbf{Scenario 2}: Line formation to pass through a horizontally narrow frame
			\item \textbf{Scenario 3}: Compact formation to pass through a small window
		\end{itemize}
		The goal is that through semantic transaction of generative models, such formations are achieved
	\end{frame}

	\begin{frame}{Potential scenarios: Vertical column avoidance}
		Unlike frames, a collective behavior is \textbf{not} necessary in this scenario
		\begin{figure}
			\centering
			\includegraphics[width=0.5\textwidth]{/home/donkarlo/Dropbox/projs/research/assets/vertical-obstacle-avoidance.png}
			\label{fig:vertical-obstacle-avoidance}
		\end{figure}
	\end{frame}

	\begin{frame}{Potential scenarios: Horizontal column avoidance}
		Unlike frames, a collective behavior is \textbf{not} necessary in this scenario
		\begin{figure}
			\centering
			\includegraphics[width=0.5\textwidth]{/home/donkarlo/Dropbox/projs/research/assets/horizental-column-avoidance.png}
			\label{fig:vertical-obstacle-avoidance}
		\end{figure}
	\end{frame}

	\begin{frame}{Potential scenarios: Horizontal column avoidance}
		This scenario can represent a set of scenarios in orientation vector between neighboring agents does not change but a collective shift is required. 
		\begin{figure}
			\centering
			\includegraphics[width=0.5\textwidth]{/home/donkarlo/Dropbox/projs/research/assets/collective-horizental-movement-for-obstacle-avoidance.png}
			\label{fig:vertical-obstacle-avoidance}
		\end{figure}   
	\end{frame}

	\begin{frame}{Communication rules}
		To keep communication decentralized, a ranking strategy according to closeness of agents to destination is needed and the following rules should be observed. 
		\begin{itemize}
			\item No agent can transmit the generative models from other agents to the neighboring agent.
			\item Messages can be made of one generative model and the senders rank or more than one (for distributional semantics) 
			\item Messages can only be transmitted to neighboring ranked nodes when the generative model an agent is practicing changes.
		\end{itemize}
	\end{frame}

	\begin{frame}{Communication rules - Ranking}
		\begin{figure}
			\centering
			\includegraphics[width=0.6\textwidth]{/home/donkarlo/Dropbox/projs/research/assets/agents-rankings-according-to-their-closeness-to-destination.png}
			\label{fig:agents-rankings-according-to-their-closeness-to-destination.png}
		\end{figure}
	\end{frame}

	\begin{frame}{Requirements}
		\begin{itemize}
			\item At least three drones so that neighboring communication is meaningful, although the results could be evaluated a lot better using more drones
			\item Minimum two different sensors to establish a relationship between heterogeneous sensors. The best of such sensors for depth and obstacle selection in low speed are active sensors: 
			\begin{itemize}
				\item Lidar
				\item Sonar
			\end{itemize}
		\end{itemize}
	\end{frame}
\end{document}
