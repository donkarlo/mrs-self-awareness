% !TeX TXS-program:bibliography = txs:///biber
\documentclass[unknownkeysallowed]{beamer}
\usetheme{UniKlu}
\usepackage[backend=biber,style=apa,sorting=nty, bibencoding=utf8]{biblatex}
\addbibresource{/home/donkarlo/Dropbox/projs/research/refs.bib}


\usepackage{xcolor}

\title{Proposal experimental plan draft - Version 0.1}
\author{Mohammad Rahmani}
\institute{DECIDE Doctoral School}

\begin{document}
	\begin{frame}
		\date{}
		\maketitle
		\textcolor{white}{\textbf{12 August 2020}}
	\end{frame}

	\begin{frame}{Proposal Goal}
		\textbf{Scientific}
		\begin{itemize}
			\item To find out whether intelligent agents semantic-awareness help with emergence of better collective behavior (NULL hypothesis)
		\end{itemize}
		
		\textbf{Technical}
		\begin{itemize}
			\item Scalable object transportation (and vertical landing in vertical mode)
		\end{itemize}
	\end{frame}

	\begin{frame}{Semantic-awareness meaning}
		\begin{itemize}
			\item A mutual understanding of the meaning of the language from which generative (Bayesian) models that agents use are derived
			\item Considering the meaning of different parts of this language in the context in which they appear to provide
				\begin{itemize}
					\item Temporality semantics
					\item Similarity semantics
				\end{itemize}
		\end{itemize}
	\end{frame}

	\begin{frame}{Goal - semantic perspective}
		Each agent receives sequences of generative models that neighboring agents have, are or will experience(d) and they decide for individual actions which emerges in a collective behavior to solve a problem.
	\end{frame}
	
	\begin{frame}{Related study area}
		\begin{itemize}
			\item Bayesian self-aware Artificial Intelligence 
			\item Collective adaptive systems
				\begin{itemize}
					\item Collective object transportation
				\end{itemize}
			\item Swarm navigation and Self-organizing
			\item Dynamic system modeling
				\begin{itemize}
					\item Dynamic Bayesian modeling
				\end{itemize}
			\item Discretization of continuous features
			\item Semantics
		\end{itemize}
	\end{frame}

	\begin{frame}{Methodology}
		A Bayesian self-aware Artificial intelligence approach will be taken which must include
		\begin{itemize}
			\item Individual perception
			\item Discretization (using clustering methods) of state space and derive the alphabet of two languages
				\begin{itemize}
					\item Alphabets of words describing interaction between neighboring agents
					\item Alphabets of words describing individual dynamism 
				\end{itemize}
			\item Abnormality detection in individual motion
		\end{itemize}
	\end{frame}

	\begin{frame}{Methodology - 2}
		\begin{itemize}
			\item Abnormality detection in interaction
			\item Generative individual models
			\item Generative interaction models
			\item Descriminative models
			\item Control decision making according abnormality detection
		\end{itemize}
	\end{frame}

	\begin{frame}{Horizontal Frame Scenarios}
		Reference transportation
		\begin{figure}
			\centering
			\includegraphics[width=0.45\textwidth]{/home/donkarlo/Dropbox/projs/research/assets/swarm-drones-reference-task.png}
			\label{fig:swarm-drones-reference-task}
		\end{figure}
		The importance of such obstacles is that every surface can be divided to small, similar surfaces such that they approach the shape of the surface. 
	\end{frame}

	\begin{frame}{Horizontal Frame Scenarios - Frame passage transportation}
		New Generative DBN models can be learned out of the following scenarios. 
		\begin{figure}
			\centering
			\includegraphics[width=0.7\textwidth]{/home/donkarlo/Dropbox/projs/research/assets/vertical-frame-scenarios.png}
			\label{fig:collective-behaviour-learning}
		\end{figure}
	\end{frame}

	\begin{frame}{Horizontal Frame Scenarios}
		New Generative DBN models can be learned out of the following scenarios. 
		\begin{figure}
			\centering
			\includegraphics[width=0.7\textwidth]{/home/donkarlo/Dropbox/projs/research/assets/vertical-frame-scenarios.png}
			\label{fig:collective-behaviour-learning}
		\end{figure}
	\end{frame}

	\begin{frame}{Horizontal Frame Scenarios}
		\begin{itemize}
			\item \textbf{Scenario 1}: Row formation to pass through a vertically narrow frame 
			\item \textbf{Scenario 2}: Line formation to pass through a horizontally narrow frame
			\item \textbf{Scenario 3}: Compact formation to pass through a small window
		\end{itemize}
		The goal is that through semantic transaction of generative models, such formations are achieved
	\end{frame}

	\begin{frame}{Scenario: Vertical column avoidance}
		Unlike frames, a collective behavior is \textbf{not} necessary in this scenario
		\begin{figure}
			\centering
			\includegraphics[width=0.5\textwidth]{/home/donkarlo/Dropbox/projs/research/assets/vertical-obstacle-avoidance.png}
			\label{fig:vertical-obstacle-avoidance}
		\end{figure}
	\end{frame}

	\begin{frame}{Scenario: Horizontal column avoidance}
		Unlike frames, a collective behavior is \textbf{not} necessary in this scenario
		\begin{figure}
			\centering
			\includegraphics[width=0.5\textwidth]{/home/donkarlo/Dropbox/projs/research/assets/horizental-column-avoidance.png}
			\label{fig:vertical-obstacle-avoidance}
		\end{figure}
	\end{frame}

	\begin{frame}{Scenario: Horizontal column avoidance}
		More scenarios could be devised such that interaction orientation changes between consecutive ranks is not necessary
		\begin{figure}
			\centering
			\includegraphics[width=0.5\textwidth]{/home/donkarlo/Dropbox/projs/research/assets/vertical-obstacle-avoidance-by-general-movement.png}
			\label{fig:vertical-obstacle-avoidance}
		\end{figure} 
	\end{frame}

	\begin{frame}{Scenario: Horizontal column avoidance}
		The goal of the last three scenarios is to prove that semantic communication not only helps better individual actions but also it can signify whether a communication is needed at all or not.    
	\end{frame}

	\begin{frame}{Communication rules}
		To keep communication decentralized, a ranking strategy according to closeness of agents to destination is needed and the following rules should be observed. 
		\begin{itemize}
			\item No agent can transmit the generative models from other agents to the neighboring agent.
			\item Messages can be made of one generative model and the senders rank or more than one (for distributional semantics) 
			\item Messages can only be transmitted to neighboring ranked nodes when the generative model an agent is practicing changes.
		\end{itemize}
	\end{frame}

	\begin{frame}{Communication rules - Ranking}
		\begin{figure}
			\centering
			\includegraphics[width=0.6\textwidth]{/home/donkarlo/Dropbox/projs/research/assets/agents-rankings-according-to-their-closeness-to-destination.png}
			\label{fig:agents-rankings-according-to-their-closeness-to-destination.png}
		\end{figure}
	\end{frame}

	\begin{frame}{Requirements}
		\begin{itemize}
			\item At least three drones so that neighboring communication is meaningful, although the results could be evaluated a lot better using more drones
			\item Minimum two different sensors to establish a relationship between heterogeneous sensors. The best of such sensors for depth and obstacle selection in low speed are active sensors: 
			\begin{itemize}
				\item Lidar
				\item Sonar
			\end{itemize}
		\end{itemize}
	\end{frame}
\end{document}
