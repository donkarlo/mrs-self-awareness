% !TeX TXS-program:bibliography = txs:///biber
\documentclass[unknownkeysallowed]{beamer}
\usetheme{UniKlu}
\usepackage[backend=biber,style=apa,sorting=nty, bibencoding=utf8]{biblatex}
\addbibresource{/home/donkarlo/Dropbox/projs/research/refs.bib}


\usepackage{xcolor}

\title{Proposal draft for experimental plan - Version 0.2}
\author{Mohammad Rahmani}
\institute{DECIDE Doctoral School}

\begin{document}
	\begin{frame}
		\date{}
		\maketitle
		\textcolor{white}{\textbf{26 August 2020}}
	\end{frame}

	\begin{frame}{}
		\textbf{Scientific goal}
		\begin{itemize}
			\item The null hypothesis will probably be testing a method derived from slide 9 (\textbf{Probable solutions}) to improve the (at least one of the) \textbf{self-awareness} challenges mentioned in Slide 3 (\textbf{Challenges}).  
		\end{itemize}
		
		
		\textbf{Technical application}
		\begin{itemize}
			\item Collective areal manipulation (load transportation) in tight corridors
		\end{itemize}
		
	\end{frame}

	\begin{frame}{Challenges}
		Based on an autonomous \textbf{self-aware} paradigm
		\begin{itemize}
			\item Motion (Path) planning (Generative Model generation)
			\item Trajectory tracking which involves
				\begin{itemize}
					\item State estimation techniques
				\end{itemize}
			\item Collision avoidance which involves
				\begin{itemize}
					\item \textbf{Major abnormal detection} (A new model is needed)
					\item Descriminative Flight mode selection
				\end{itemize}
			\item Disturbance rejection (dealing with minor abnormalities)
				\begin{itemize}
					\item Minor abnormally detection (Current Trajectory needs to be modified)
					\item Decision making and control 
				\end{itemize}
			\item ...
		\end{itemize}
		There are existing solutions for all aforementioned problems in individual drones and drone swarms but not necessarily in cooperative areal payload transportation. 
	\end{frame}

	\begin{frame}{}
		content
	\end{frame}

	\begin{frame}{Scenarios - Reference}
		Rigid load transportation from one location to another by tracking a straight line at a fixed height 
		\begin{figure}
			\centering
			\includegraphics[width=0.5\textwidth]{/home/donkarlo/Dropbox/projs/research/assets/reference-collective-load-transportation-task.png}
		\end{figure}
	\end{frame}

	\begin{frame}{Scenarios - Horizontal collision avoidance}
		\begin{figure}
			\centering
			\includegraphics[width=0.35\textwidth]{/home/donkarlo/Dropbox/projs/research/assets/collective-horizontal-collision-avoidance.png}
		\end{figure}
	\end{frame}

	\begin{frame}{Scenarios - Vertical collision avoidance}
		\begin{figure}
			\centering
			\includegraphics[width=0.35\textwidth]{/home/donkarlo/Dropbox/projs/research/assets/collective-vertical-collision-avoidance.png}
		\end{figure}
	\end{frame}

	\begin{frame}{Scenarios - Turning}
		\begin{figure}
			\centering
			\includegraphics[width=0.6\textwidth]{/home/donkarlo/Dropbox/projs/research/assets/collective-rigid-payload-transportation-in-turning-point.png}
		\end{figure}
	\end{frame}

	\begin{frame}{Scenarios - Altogether scenario}
		Together, all the following three scenarios must make payload transportation possible in tight corridors
		\begin{figure}
			\centering
			\includegraphics[width=0.6\textwidth]{/home/donkarlo/Dropbox/projs/research/assets/collective-rigid-load-transportation-in-tight-corridors.png}
		\end{figure}
	\end{frame}
	

	\begin{frame}{Probable solutions to improve the state of the art}
		\begin{itemize}
			\item Swarm intelligence to model emergence of a collective behavior from local, agents behavior
			\item Local Communication through sharing practicing generative models with neighbors using 
				\begin{itemize}
					\item An individual agent state description language with words made of alphabets of different derivatives of time
					\item A collective state description language with words made of alphabets of possible co-occurrence of different generalized state space regions  
				\end{itemize}
			\item Formation control which involve studying architecture such as 
			\begin{itemize}
				\item Static Leader-follower architecture
				\item Dynamic Leader Follower architecture
				\item Leaderless architecture
				\item ...
			\end{itemize}
			\item ...
		\end{itemize}
	\end{frame}
\end{document}
