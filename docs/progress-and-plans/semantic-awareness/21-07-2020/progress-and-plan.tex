% !TeX TXS-program:bibliography = txs:///biber
\documentclass[unknownkeysallowed]{beamer}
\usetheme{UniKlu}
\usepackage[backend=biber,style=apa,sorting=nty, bibencoding=utf8]{biblatex}
\addbibresource{/home/donkarlo/Dropbox/projs/research/refs.bib}


\usepackage{xcolor}

\title{Collective Self-awareness in Progress and Plans}
\author{Mohammad Rahmani}
\institute{DECIDE Doctoral School}

\begin{document}
	\begin{frame}
		\maketitle
	\end{frame}
	
	\begin{frame}{SA}
		\begin{itemize}
			\item Initialization
		\end{itemize}
	\end{frame}

	\begin{frame}{Collective SA}
		\begin{itemize}
			\item Initialization
		\end{itemize}
	\end{frame}

	\begin{frame}{Semantic SA - Activated Area}
		\textbf{Training phase}
		\begin{itemize}
			\item Word level
			\item Sentence level
		\end{itemize}
	\end{frame}

	\begin{frame}{Semantic SA - Sharing Models}
		\textbf{Training phase}
		\begin{itemize}
			\item Word are built by clustering 
		\end{itemize}
	\end{frame}

	\begin{frame}{Future plans}
		Building words on sensor heterogeneity level
		\begin{itemize}
			\item Different sensors (e.g one agent Temperature)
		\end{itemize}
		Goals and Path
		\begin{itemize}
			\item Making vocabulary between data of different sensors
		\end{itemize}
	\end{frame}

	\begin{frame}{Methods to influence a decision}
		\begin{itemize}
			\item Declaring intention by sentences (sequence of words)
			\item Describing situation by sentences (sequence of words)
		\end{itemize}
	\end{frame}


	\begin{frame}{Probable plan for Force field common activated area}
		\textbf{Force Field}
		First act according to sensors then assess inform intention based on common formed words:
		\begin{itemize}
			\item Reinforcement learning
				\begin{itemize}
					\item Rewards will be modeled by attractive forces
					\item Costs will be modeled by repulsive forces
				\end{itemize}
		\end{itemize}
		\textbf{Vocabulary}
		\begin{itemize}
			\item Must be formed among different types of sensors 
		\end{itemize}
	\end{frame}

	\begin{frame}{Probable plan for semantic awareness}
		\begin{itemize}
			\item Similar word in a different sequences must induce new actions
		\end{itemize}
		Such as in driving
		\begin{itemize}
			\item Taking over
				\begin{itemize}
					\item Attractor vehicle  = Turning left + increase speed + Turning right
					\item Follower vehicle = nothing + decrease speed a little + nothing
				\end{itemize}
			\item Getting out of park:
				\begin{itemize}
					\item Attractor vehicle = Turn left + increase speed
					\item Follower vehicle =  move forward a little
				\end{itemize}
		\end{itemize}
	\end{frame}
	
	
	\begin{frame}[allowframebreaks]{References}
		\printbibliography
	\end{frame}
\end{document}
