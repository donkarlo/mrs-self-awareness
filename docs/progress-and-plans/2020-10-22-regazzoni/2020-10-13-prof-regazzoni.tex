% !TeX document-id = {412070ce-2b89-4575-900e-e9d3183d0ba4}
% !TeX TXS-program:bibliography = txs:///biber
\documentclass[unknownkeysallowed]{beamer}
\usetheme{UniKlu}
\usepackage[backend=biber,style=apa,sorting=nty, bibencoding=utf8]{biblatex}
\addbibresource{/home/donkarlo/Dropbox/projs/research/refs.bib}


\usepackage{xcolor}

\title{From Individual Perception to Collective Behavior in MAV. A self-aware approach}
\author{Mohammad Rahmani}
\institute{Pervasive Computing Group}

\begin{document}
	\begin{frame}
		\date{}
		\maketitle
		\textcolor{white}{\textbf{6 October 2020}}
	\end{frame}
	
	\begin{frame}{Swarms}
		Swarms a set of identical robots that
		\begin{itemize}
			\item locally perceive via sensors
			\item locally act through actuators
			\item locally communicate
		\end{itemize}
		but are expected to 
		\begin{itemize}
			\item behave collectively
		\end{itemize}
	\end{frame}

	\begin{frame}{An approach in collective awareness}
	Imagine a collection of agents navigating from a starting point to a destination (attraction) point and on their way they face repulsive forces against which they should overcome
		\begin{itemize}
			\item As the dynamism of each IA can be described in the form of a DBN, the course of their relationship over time until reaching the destination can also be described by a DBN
		\end{itemize}
	\end{frame}

	\begin{frame}{An approach in collective awareness}
		\textbf{Formation}: Is a consensus between a set agents to maintain the distance vector between themselves over a certain amount of time i.e. if $A$ is the set of the agents in the system
		\begin{equation}
		A = \{a_1,...,a_{|A|}\}
		\end{equation}
		then the distance vector to form the formation $i$ that agent $a_j$ should be aware about $k$ neighboring agents cab be defined as 
		\begin{equation}
		D_{ijk} = \{\vec{d}_{jp}(a_j,a_p)|p\in \{1,...,n_k\}\} 
		\end{equation}
		The formation $i$ for the whole system can be defined
		\begin{equation}
		F_i = \{D_{ij}|i\in\{1,...,|A|\}\}
		\end{equation}
		If a formation is maintained for time $t$ then $F_t$ will present a temporal formation. 
	\end{frame}

	\begin{frame}{Temporal formation}
		\begin{figure}
			\includegraphics[scale=0.7]{/home/donkarlo/Dropbox/projs/research/assets/collective-frame-passing-of-three-drones.jpg}
			\caption{}
		\end{figure}
	\end{frame}

	\begin{frame}{Maneuver, Collective behavior}  
		\begin{itemize}
			\item \textbf{Maneuver} is changing from one formation to another over the course of time.
			\item \textbf{Collective behavior or a temporal behavior} is a temporal formation or a maneuver between the agents. 
		\end{itemize}
	\end{frame}

	\begin{frame}{Maneuver}
		\begin{figure}
			\includegraphics[scale=0.7]{/home/donkarlo/Dropbox/projs/research/assets/collective-behavior-of-three-drones-to-pass-a-narrow-passage.jpg}
			\caption{}
		\end{figure}
	\end{frame}

	\begin{frame}{Predictive models for collective behavior}
		\begin{itemize}
			\item Both maintaining a temporal formation or maneuver are dynamic systems which shows the \textbf{evolution of distance vector over the course of time} and hence could be formalized by a transition matrix. 
			\item Can this transition matrix trained by repeating the temporal formation or the maneuver and be stored for future state predictions as a \textbf{model} or \textbf{experience} as a DBN simlar to an individual agent state prediction model?
		\end{itemize}
	\end{frame}

	\begin{frame}{Training the collective behavior DBN}
		\begin{itemize}
			\item One approach is to \textbf{discretize} the \textbf{state space} of each individual agent in a collective behavior to \textbf{quasi-motion constant regions} and the temporal sequence of co-occurrence of each agent's region with another neighboring agent can form the alphabet for the words of a sentence describing the evolution of the relation between two agents. (Taking an approach similar to \footnote{Baydoun 2019 Prediction of multi target dynamics using discrete descriptors an interactive approach})
			\item An \textbf{anomaly} can be defined as the \textbf{distance} above the \textbf{tolerance rate} between \textbf{expected/predicted state} in this relationship and \textbf{observed state} in comparison to practicing model.
		\end{itemize}
	\end{frame}

	\begin{frame}{Why should we observe anomaly in a collective behavior?}
		\begin{itemize}
			\item Each agent tries to \textbf{save itself} as such the intersecting quasi-motion regions will not occur as they were predicted and collective anomaly occurs. 
			\item But if other agents do not react accordingly the whole system collapse (Systems homeostasis will collapse for the sake of maintaining an individual's homeostasis).  
			\item \textbf{Solution}: Choosing the right collective DBN to practice according to which each individual should adjust it's motion to avoid whole system's collapse
		\end{itemize}
	\end{frame}

	\begin{frame}{Probable implementation}
		\begin{enumerate}
			\item  \textbf{Discretizing} the motion state space of agents to \textbf{different derivative orders of time} \footnote{Kanapram 2020 Collective awareness for abnormality detection in connected autonomous vehicles}
			\item Use the \textbf{composition} of these \textbf{different derivative orders} as \textbf{words} to communicate any \textbf{motion state change} of each agent to neighboring agents
			\item 	Train a model to \textbf{map} the \textbf{words} \textbf{received} by an agent from its neighboring agents to a \textbf{collective behavior} defined by the words of the DBN representing a collective behavior. 
		\end{enumerate}
	\end{frame}

	\begin{frame}{Scenarios}
		\begin{figure}
			\includegraphics[scale=0.5]{/home/donkarlo/Dropbox/projs/research/assets/scenarios-of-collective-drones.jpg}
			\caption{}
		\end{figure}
	\end{frame}
\end{document}
