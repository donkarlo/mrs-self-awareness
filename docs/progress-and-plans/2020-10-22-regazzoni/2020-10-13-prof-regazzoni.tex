% !TeX document-id = {412070ce-2b89-4575-900e-e9d3183d0ba4}
% !TeX TXS-program:bibliography = txs:///biber
\documentclass[unknownkeysallowed]{beamer}
\usetheme{UniKlu}
\usepackage[backend=biber,style=apa,sorting=nty, bibencoding=utf8]{biblatex}
\addbibresource{/home/donkarlo/Dropbox/projs/research/refs.bib}


\usepackage{xcolor}

\title{From Individual Perception to Collective Behavior in MAV. A self-aware approach}
\author{Mohammad Rahmani}
\institute{Pervasive Computing Group}

\begin{document}
	\begin{frame}
		\date{}
		\maketitle
		\textcolor{white}{\textbf{23 October 2020}}
	\end{frame}
	
	
	\begin{frame}{Swarms}
		Swarms are teams of homogeneous robots that
		\begin{itemize} 
			\item locally perceive via sensors
			\item locally (re)act through actuators
			\item locally communicate
		\end{itemize}
		but are expected to 
		\begin{itemize}
			\item behave collectively
		\end{itemize}
	to accomplish a task. (E.g collective payload delivery)
	\end{frame}

	\begin{frame}{Why swarms?}
		\begin{itemize}
			\item \textbf{Improved performance}: if tasks can be decomposable then by using parallelism, groups can make tasks to be performed more efficiently.
			\item \textbf{Task enablement}: groups of robots can do certain tasks that are impossible for a single robot.
			\item \textbf{Distributed sensing}: the range of sensing of a group of robots is wider than the range of a single robot.
			\item \textbf{Distributed action}: a group a robots can actuate in different places at the same time.
			\item \textbf{Fault tolerance}: under certain conditions, the failure of a single robot within a group does not imply that the given task cannot be accomplished, thanks to the redundancy of the system.
		\end{itemize}
	\end{frame}

	\begin{frame}{Leadership in swarms}
		\begin{itemize}
			\item Static leadership
			\item Dynamic leadership (E.g. In collective movement: The agent closer to the destination takes temporarily the leadership)
			\item No leader (Most frequent in swarms. E.g. Each agent knows the final destination)
		\end{itemize}
	\end{frame}

	\begin{frame}{Issues in maintaining collective homeostasis conditions}
		The ultimate goal of self-awareness is to help an IA to maintain its homeostasis condition. One approach to achieve this is to equip an IA with
		\begin{itemize}
			\item an initial experience with which it can achieve its task
			\item the ability to detect new experiences
				\begin{itemize}
					\item model, store and retrieve them back for prediction 
				\end{itemize}
			\item make appropriate decisions according to future states predictions
		\end{itemize}
	\end{frame}

	\begin{frame}{Issues in maintaining collective homeostasis conditions}
		The same can be applied to collective awareness which translates to anomaly detection in existing course of relation between agents. This course of relation can be formalized as the dynamism of vector distance between agents. (But there are other ways too\footfullcite{baydoun-2019-prediction-of-multi-target-dynamics-using-discrete-descriptors-an-interactive-approach}) 
		
		\begin{figure}
			\includegraphics[scale=0.4]{/home/donkarlo/Dropbox/projs/research/assets/collective-3-payload.png}
			\caption{}
		\end{figure}
	\end{frame}
	\begin{frame}{Issues in maintaining collective homeostasis conditions}
		Example: Flocking (Reynolds laws of boids: how to stay in a flock)
		\begin{itemize}
			\item Move in the same direction as their neighbors
			\item Alignment - steer towards average heading of neighbors
			\item Cohesion - steer towards average position of neighbors (long range attraction)
		\end{itemize}
	\end{frame}

	\begin{frame}{Issues in maintaining collective homeostasis conditions}
		Example: Collective load transportation
		\begin{itemize}
			\item If the two drones are too close, then the whole system's homeostasis level is threatened by external disturbances
			\item If the two drones are too far away from each other then the two drones higher energy consumption threatens the homeostasis level of the system
		\end{itemize}
		\begin{figure}
			\includegraphics[scale=0.5]{/home/donkarlo/Dropbox/projs/research/assets/two-drones-homeo-staisis-collective-behavior.png}
			\caption{}
		\end{figure}
	\end{frame}
	
	\begin{frame}{Dynamic Individual behavior modeling, reference task}
		\begin{figure}
			\includegraphics[scale=0.5]{/home/donkarlo/Dropbox/projs/research/assets/initial-single-dron-load.png}
			\caption{}
		\end{figure}
	\end{frame}

	\begin{frame}{Modeling new individual experiences}
		\begin{figure}
			\includegraphics[scale=0.5]{/home/donkarlo/Dropbox/projs/research/assets/ind-tasks.jpg}
			\caption{}
		\end{figure}
	\end{frame}

	\begin{frame}{Modeling new individual experiences}
		 \cite{kanapram-2020-collective-awareness-for-abnormality-detection-in-connected-autonomous-vehicles}\footfullcite{kanapram-2020-collective-awareness-for-abnormality-detection-in-connected-autonomous-vehicles} uses words formed of different time derivative of states to train the individual DBNs.
		 \begin{equation}
		 	W_j = \{\alpha^{(0)},...,\alpha^{(L)}\}
		 \end{equation}
	\end{frame}

	\begin{frame}{Dynamic collective behavior modeling - reference task}  
		\begin{figure}
			\includegraphics[scale=0.45]{/home/donkarlo/Dropbox/projs/research/assets/initial-multi-drones-load.png}
			\caption{}
		\end{figure}
	\end{frame}

	\begin{frame}{Dynamic new collective behavior modeling}  
		\begin{figure}
			\includegraphics[scale=0.45]{/home/donkarlo/Dropbox/projs/research/assets/collective-dbns.jpg}
			\caption{}
		\end{figure}
	\end{frame}

	\begin{frame}{Modeling new collective experiences - Training}
		Each experience can be modeled as mutual quasi-constant state regions proposed in \cite{baydoun-2019-prediction-of-multi-target-dynamics-using-discrete-descriptors-an-interactive-approach}\footfullcite{baydoun-2019-prediction-of-multi-target-dynamics-using-discrete-descriptors-an-interactive-approach} 
		\begin{equation}
		W_{ij} = \{S_i,S_j\}
		\end{equation}
		\begin{equation}
		DBN_{f_{ij}} = \{D_{f_{1_{ij}}},...,D_{f_{n_{ij}}}\}
		\end{equation}
		For each experience we have one collective experience
		\begin{equation}
		DBNs = \{DBN_1,...,DBN_p\}
		\end{equation}
	\end{frame}

	\begin{frame}{Question}
		If each IA communicate its current state to its neighbors with \cite{kanapram-2020-collective-awareness-for-abnormality-detection-in-connected-autonomous-vehicles} words, how precisely can we map a set of communicated words to a collective DBN in \cite{baydoun-2019-prediction-of-multi-target-dynamics-using-discrete-descriptors-an-interactive-approach} or recognize it as a new collective experience. 
		\begin{itemize}
			\item In other words can informing on individual efforts to maintain individual homeostasis help with a quicker decision of practicing an appropriate collective behavior?
		\end{itemize}
		\begin{figure}
			\includegraphics[scale=0.45]{/home/donkarlo/Dropbox/projs/research/assets/saving.png}
			\caption{}
		\end{figure}
	\end{frame}
\end{document}
