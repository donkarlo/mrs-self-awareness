% !TeX TXS-program:bibliography = txs:///biber
\documentclass[unknownkeysallowed]{beamer}
\usetheme{UniKlu}
\usepackage[backend=biber,style=apa,sorting=nty, bibencoding=utf8]{biblatex}
\addbibresource{refs.bib}


\usepackage{xcolor}

\title{Collective Self-awareness in Multi-Robot Systems}
\author{Mohammad Rahmani}
\institute{DECIDE Doctoral School}

\begin{document}
\begin{frame}
	\maketitle
\end{frame}


\begin{frame}{Biological Intelligent Agent (IA)}
	Every intelligent agent such as human has the ability to relate what is happening outside to what is happening inside it.
	These events are perceivable by sensors. Outside could be observed by sensors such as eye and inside could be perceived by sensors such as cochlea
\end{frame}

\begin{frame}{Biological IA}
	\begin{itemize}
		\item Initial knowledge
		\item 
	\end{itemize}
$DBN_1$

Human brain and the choice of between $DBNs$ -  free energy
\end{frame}

\begin{frame}{In single IA}
Multi layer DBNs
One maps from visual data (extroceptive) to proporioceptive 
One approach: control data will be observation and position will be real states
Now what we need to do is 
1. To model the relationship between control data (observation/evidence/proprioceptive) and position (continuous states/proprioceptive)

the first layer says if you apply this amount of power/velocity/steering etc while you are position $X_t$ then you expect to be at position $X_{t+1}$ 

2. To map the states to compos-able semantics - to describe new situations based on previous situations



The reason to form different strategies for temporal, cause-effect super/semantic states is to enable the IA to make a distinction between meaningful and un-meaningful temporal, cause-effect sequences. 
\end{frame}


\begin{frame}{In Collection of IAs}
The relation
Changes in the distance vector could have been studied. But a general solution can the co-occurence of semantic states
\end{frame}

\begin{frame}[allowframebreaks]{References}
	\printbibliography
\end{frame}
\end{document}
