\documentclass{article}
\usepackage{comment}
\usepackage[english]{babel}
\usepackage[utf8]{inputenc}
\usepackage{fancyhdr}
\usepackage[round]{natbib}
\usepackage{graphicx}
\usepackage{url}
\usepackage{amsmath}
\usepackage{amssymb}
\DeclareMathOperator*{\argmax}{argmax}
\DeclareMathOperator*{\argmin}{argmin}
\pagenumbering{arabic}
\usepackage{multicol}
\usepackage{siunitx}

\pagestyle{fancy}
\fancyhf{}
\rhead{Mohammad Rahmani}
\lhead{CA in MRS}

\newcommand{\ignore}[1]{}
\begin{document}
	\bibliographystyle{plainnat}
	\title{A Survey of Collective self-awareness (CA) in Multi-robot Systems (MRS)}
	\author{Mohammad Rahmani}
	\date{}
	\maketitle
	\cite{kanapram-2020-collective-awareness-for-abnormality-detection-in-connected-autonomous-vehicles} has proposed a DBN based approach to make two automatic cars following each other, aware of abnormalities to avoid collision. 
	
	\cite{kephart-2017-self-adaptation-in-collective-self-aware-computing-systems}
	
	\url{https://en.wikipedia.org/wiki/Collective_consciousness} 
	Bio-inspired autobiographical memories have already been investigated towards implementing self-awareness in artificial agents, for example, in\cite{landauer-2015-designing-cooperating-self-improving-systems}.
	
	\section{Self-aware/Consciousness Computational General Models}
	\citet{regazzoni-2020-multi-sensorial-generative-and-descriptive-self-awareness-models-for-autonomous-systems} in Section IV, part B (Page 19) has proposed a comprehensive model for interaction between two AVs which addresses initialization to model creation. 
	
	\citet{williams-2019-a-model-for-human-artificial-collective-consciousness-part-1} and \citet{williams-2019-a-model-for-human-artificial-collective-consciousness-part-2} borrowed the functional modeling approach common in systems and software engineering, an implementable model of the functions of human consciousness proposed to have the capacity for general problem solving ability transferable to any domain, or true self-aware intelligence, is presented. Its functional model is independent of implementation and proposed to also be applicable to artificial consciousness, and to platforms that organize individuals into what is defined here as a first order collective consciousness, or at higher orders into what is defined here as Nth order collective consciousness.
	
	\citet{esterle-2020-i-think-therefore-you-are-models-for-interaction-in-collectives-of-self-aware-cyber-physical-systems} extends the idea of the self-awareness of individual systems toward networked self-awareness. This gives systems the ability to reason about how they are being affected by the actions and interactions of others within their perceived environment, as well as in the extended environment that is beyond their direct perception. They propose that different levels of networked self-awareness can develop over time in systems as they do in humans. Furthermore, they propose that this could have the same benefits for networks of systems that it has had for communities of humans, increasing performance and adaptability.
	
	\cite{celentano-2016-multi-robot-systems-machine-machine-and-human-machine-interaction-and-their-modelling} suggests an interworking cognitive entities model which includes explicitly interworking capabilities and is applied to both machine-machine interaction and human-machine interaction.
	
	
	\cite{diaconescu-2017-architectures-for-collective-self-aware-computing-systems}
	\cite{gerasimou-2019-towards-systematic-engineering-of-collaborative-heterogeneous-robotic-systems} suggests a future vision toward coupling heterogeneous to develop collective self awareness such that they can assist each other in accomplishing tasks.
	
	\cite{kosak-2019-multipotent-systems-combining-planning-self-organization-and-reconfiguration-in-modular-robot-ensembles} they presented an approach to filling the gap between heterogeneous and homogeneous robots by introducing a reference architecture for mobile robots that defines the interplay of all necessary technologies for achieving this goal. They introduce the class of robot systems implementing this architecture as multipotent systems that bring together the benefits of both system classes, enabling homogeneously designed robots to become heterogeneous specialists at runtime.
	
	
	
	\section{Interaction}
	\cite{baydoun-2020-prediction-of-multi-target-dynamics-using-discrete-descriptors-an-interactive-approach} proposes a probabilistic method to track and interpret the interactions of moving objects.
		
	\section{Self-reconfiguration}
		\cite{pena-2019-blockchain-powered-collaboration-in-heterogeneous-swarms-of-robots} hs implemented a model on a swarm of drones to address the complexity of data generated by multi robots systems functioning in an environment. They  brought elastic computing techniques and dynamic resource management from the edge-cloud computing domain to the swarm robotics domain. This enables the dynamic provisioning of collective capabilities in the swarm for
		different applications. Therefore, we transform a swarm into a distributed sensing and computing platform capable of complex data processing tasks, which can then be offered as a service.
	
	
	
	\section{Nodes and networks}
		\cite{agne-2016-self-aware-compute-nodes}
	
	\section{Definition}
		“Information about the global state of the system, which feeds back to adaptively control the actions of the system’s low-level components. This information about the global state is distributed and statistical in nature, and thus is difficult for observers to tease out. However, the system’s components are able, collectively, to use this information in such a way that the entire system appears to have a coherent and useful sense of its own state \citep{mitchell-2005-self-awareness-and-control-in-decentralized-systems}. \cite{schmickl-2011-cocoro-the-self-aware-underwater-swarm} showed that a group of robots with simple behavioral rules
		and local interactions may achieve collective awareness of a global state, distributed across the individual units.
		
		\paragraph{} The emphasis here has been added, to highlight that a system which behaves in a self-aware manner is not necessarily required to possess a single component which has access to system global knowledge.Indeed, in many cases, e.g., ant colonies, immune systems and humans themselves, the entire system appears self-aware, despite the knowledge available at constituent parts being only local. The appearance of self-awareness is an emergent effect \citet{mitchell-2005-self-awareness-and-control-in-decentralized-systems}.
		\paragraph{} This is a key observation which can contribute to the design of self-aware systems: one need not require that such a system possesses a global omniscient controller. Indeed, many natural systems appear to have been favored
		by evolution which do not have such a central point of control, and rely upon relevant knowledge being available
		at required locations within the system. It is highly likely that this can improve the robustness and adaptability of
		such systems; these are desirable properties for natural and artificial systems alike
		
		
		\citet{mitchell-2005-self-awareness-and-control-in-decentralized-systems}.
		
		\paragraph{WORKING DEFINITION FOR SELF-AWARE COMPUTING SYSTEMS}
		This definition is based on the idea of a conceptual component called a self-aware node. A node in this context need not physically exist as a hardware or software component of a computing system, but provides a conceptualisation of locality within a global system, particularly in relation to what is considered self in the context of self-awareness. This distributed nature of conceptual components is particularly relevant to the idea of distributed self-awareness, as expounded by \citet{mitchell-2005-self-awareness-and-control-in-decentralized-systems}. The definition is as follows.
		\\
		To be self-aware a node must:
		\begin{itemize}
			\item Possess information about its internal state
			(private self-awareness).
			\item Possess sufficient knowledge of its environment to determine how it is perceived by other
			parts of the system (public self-awareness).
		\end{itemize}
		Optionally, it might also:
		\begin{itemize}
			\item Possess knowledge of its role or importance
			within the wider system.
			\item Possess knowledge about the likely effect of
			potential future actions / decisions.
			\item Possess historical knowledge.
			\item Select what is relevant knowledge and what is
			not.
		\end{itemize}
	\section{Self-collision avoid}
	\cite{selvaggio-2017-towards-a-self-collision-aware-teleoperation-framework-for-compound-robots} lays the foundations of a self-collision aware teleoperation framework for compound robots. Their objective of the proposed system is to constrain the user to teleoperate a slave robot inside its safe workspace region through the application of force cues on the master side of the bilateral teleoperation system.
	\cite{kaiser-2020-towards-self-aware-multirotor-formations} proposes a framework to combine self-aware computing with multirotor formations to address this problem. The self-awareness is envisioned to improve the
	dynamic behavior of multirotors. The formation scheme that is implemented is called platooning,
	which arranges vehicles in a string behind the lead vehicle and is proposed to bring order into chaotic
	air space.
	\section{In nature}
	\citet{mitchell-2005-self-awareness-and-control-in-decentralized-systems} discusses how fish, bees, ants, immune systems form collective self-awareness in nature. 	
	
	\section{The rest}
	\cite{kernbach-2011-awareness-and-self-awareness-for-multi-robot-organisms}
	 
	CoCoRo - The Self-aware Underwater Swarm \cite{schmickl-2011-cocoro-the-self-aware-underwater-swarm}
	\section{Surveys}
	\cite{lewis-2011-a-survey-of-self-awareness-and-its-application-in-computing-systems} starting from section II-C.
	
	\section{Upcoming Seminar}
	\url{https://proceedingsoftheieee.ieee.org/connecting-the-past-and-future/webinar-series/}
	\bibliography{/home/donkarlo/Dropbox/projs/research/refs}
\end{document}