\documentclass{article}
\usepackage{comment}
\usepackage[english]{babel}
\usepackage[utf8]{inputenc}
\usepackage{fancyhdr}
\usepackage[round]{natbib}
\usepackage{graphicx}
\usepackage{url}
\usepackage{amsmath}
\usepackage{amssymb}
\DeclareMathOperator*{\argmax}{argmax}
\DeclareMathOperator*{\argmin}{argmin}
\pagenumbering{arabic}
\usepackage{multicol}
\usepackage{siunitx}

\pagestyle{fancy}
\fancyhf{}
\rhead{Mohammad Rahmani}
\lhead{Colective self-awareness in MRS}

\newcommand{\ignore}[1]{}
\begin{document}
	\bibliographystyle{plainnat}
	\title{A survey of Collective self-awareness in multiple robot systems}
	\author{Mohammad Rahmani}
	\date{}
	\maketitle
	\section{collective self-awareness}
		\cite{diaconescu-2017-architectures-for-collective-self-aware-computing-systems}
		\cite{kephart-2017-self-adaptation-in-collective-self-aware-computing-systems}
		\url{https://en.wikipedia.org/wiki/Collective_consciousness}
	\section{Nodes and networks}
		\cite{agne-2016-self-aware-compute-nodes}
	\section{Definition}
		“Information about the global state of the system, which feeds back to adaptively control the actions of the system’s low-level components. This information about the global state is distributed and statistical in nature, and thus is difficult for observers to tease out. However, the system’s components are able, collectively, to use this information in such a way that the entire system appears to have a coherent and useful sense of its own state \citep{mitchell-2005-self-awareness-and-control-in-decentralized-systems}. \cite{schmickl-2011-cocoro-the-self-aware-underwater-swarm} showed that a group of robots with simple behavioral rules
		and local interactions may achieve collective awareness of a global state, distributed across the individual units.
		
		\paragraph{} The emphasis here has been added, to highlight that a system which behaves in a self-aware manner is not necessarily required to possess a single component which has access to system global knowledge.Indeed, in many cases, e.g., ant colonies, immune systems and humans themselves, the entire system appears self-aware, despite the knowledge available at constituent parts being only local. The appearance of self-awareness is an emergent effect \citet{mitchell-2005-self-awareness-and-control-in-decentralized-systems}.
		\paragraph{} This is a key observation which can contribute to the design of self-aware systems: one need not require that such a system possesses a global omniscient controller. Indeed, many natural systems appear to have been favored
		by evolution which do not have such a central point of control, and rely upon relevant knowledge being available
		at required locations within the system. It is highly likely that this can improve the robustness and adaptability of
		such systems; these are desirable properties for natural and artificial systems alike \citet{mitchell-2005-self-awareness-and-control-in-decentralized-systems}.
		\paragraph{WORKING DEFINITION FOR SELF-AWARE COMPUTING SYSTEMS}
		This definition is based on the idea of a conceptual component called a self-aware node. A node in this context need
		not physically exist as a hardware or software component
		of a computing system, but provides a conceptualisation of
		locality within a global system, particularly in relation to
		what is considered self in the context of self-awareness.
		This distributed nature of conceptual components is particularly relevant to the idea of distributed self-awareness, as
		expounded by \citet{mitchell-2005-self-awareness-and-control-in-decentralized-systems}. The definition is as follows.
		\\
		To be self-aware a node must:
		\begin{itemize}
			\item Possess information about its internal state
			(private self-awareness).
			\item Possess sufficient knowledge of its environment to determine how it is perceived by other
			parts of the system (public self-awareness).
		\end{itemize}
		Optionally, it might also:
		\begin{itemize}
			\item Possess knowledge of its role or importance
			within the wider system.
			\item Possess knowledge about the likely effect of
			potential future actions / decisions.
			\item Possess historical knowledge.
			\item Select what is relevant knowledge and what is
			not.
		\end{itemize}
	\section{Self expression}
	\citet{lewis-2011-a-survey-of-self-awareness-and-its-application-in-computing-systems}
	\section{In nature}
	 fish, bees, ants, immune systems	
	\section{The rest}
	\cite{kernbach-2011-awareness-and-self-awareness-for-multi-robot-organisms}
	\cite{selvaggio-2017-towards-a-self-collision-aware-teleoperation-framework-for-compound-robots}
	\cite{celentano-2016-multi-robot-systems-machine-machine-and-human-machine-interaction-and-their-modelling}
	CoCoRo - The Self-aware Underwater Swarm \cite{schmickl-2011-cocoro-the-self-aware-underwater-swarm}
	\section{Surveys}
	\cite{lewis-2011-a-survey-of-self-awareness-and-its-application-in-computing-systems} starting from section II-C.
	\bibliography{/media/donkarlo/Elements/projs/research/refs}
\end{document}