% !TeX TXS-program:bibliography = txs:///biber
\documentclass[unknownkeysallowed]{beamer}
\usetheme{UniKlu}
\usepackage[backend=biber,style=apa,sorting=nty, bibencoding=utf8]{biblatex}
\addbibresource{/home/donkarlo/Dropbox/projs/research/refs.bib}


\usepackage{xcolor}

\title{NES Kickoff meeting}
\author{Mohammad Rahmani}
\institute{DECIDE Graduate School}

\begin{document}
\begin{frame}
	\maketitle
\end{frame}


\begin{frame}{General Research Studies}
	General papers were reviewed in the following research areas:
	\begin{itemize}
		\item Main area: Semantic interaction and collective semantic-awareness in self-aware MRS
			\begin{itemize}
				\item \textbf{Self-awareness (SA)}: Temporal-causal inference from contextually couple internal and external sensory data of a robot
				\item \textbf{Collective SA (CA)}: Temporal-causal inference while the internal sensory data of one robot is used as the external sensory data of another robot
				\item \textbf{Semantic interaction in CA}: The aforementioned coupled data is clustered such that each class can be considered as a letter and the inference and model creation of future states must be done from composition of such letters.
			\end{itemize}
	\end{itemize}
\end{frame}

\begin{frame}{General SA Tasks}
	\begin{itemize}
		\item \textbf{Initialization}
		\item \textbf{Memorization}
		\item \textbf{Abnormality detection}
		\item \textbf{Model creation}
		\item \textbf{Decision making}
	\end{itemize}
\end{frame}

\begin{frame}{General SA Tasks - Initialization}
	Two possible types of initialization:
	\begin{itemize}
		\item Training DBNs/ Switching DBNs (in Collective SA) using human training data. Exemplary reviewed paper \fullcite{kanapram-2020-collective-awareness-for-abnormality-detection-in-connected-autonomous-vehicles}
	\end{itemize}
	\textbf{Future plans}
	\begin{itemize}
		\item Checking for application of random walks for initialization and self-modeling.
	\end{itemize}
\end{frame}

\begin{frame}{General SA Tasks - Anomaly detection - 1}
	Based on the distance between observation and predicted state
	\begin{itemize}
		\item Predicting models (Reviewed papers)
			\begin{itemize}
				\item \textbf{Kalman Filtering (KF)} \fullcite{simon-2010-kalman-filtering-with-state-constraints-a-survey-of-linear-and-nonlinear-algorithms}
				\item \textbf{Particle Filtering (PF)} \fullcite{gustafsson-2010-particle-filter-theory-and-practice-with-positioning-applications}
				\item \textbf{Markov Jump Linear System (MJLS)} \fullcite{doucet-2001-particle-filters-for-state-estimation-of-jump-markov-linear-systems}
			\end{itemize}
	\end{itemize}
\end{frame}

\begin{frame}{General SA Tasks - Anomaly detection - 2}
	\begin{itemize}
		\item Predicting models (Read papers)
		\begin{itemize}
			\item \textbf{Markov Jump Particle Filter (MJPF)} First time appeared in: \fullcite{baydoun-2018-learning-switching-models-for-abnormality-detection-for-autonomous-driving} and was used in 
			\begin{itemize}
				\item Further exemplary papers \fullcite{regazzoni-2020-multi-sensorial-generative-and-descriptive-self-awareness-models-for-autonomous-systems}
			\end{itemize}
		\end{itemize}
	\end{itemize}
	Future Plans:
	\begin{itemize}
		\item Finding and studying more discrete-time dynamic models
	\end{itemize}
\end{frame}

\begin{frame}{General SA Tasks - Anomaly detection - 3}
	Metric system to measure the distance between predictions and observations. Determines whether a new class/letter should be created. 
	\begin{itemize}
		\item \textbf{Hellinger} \fullcite{lourenzutti-2014-the-hellinger-distance-in-multicriteria-decision-making-an-illustration-to-the-topsis-and-todim-methods}
	\end{itemize}
	Future plans, other metrics
	\begin{itemize}
		\item Bhattacharya distance
		\item JensenShannon divergence 
		\item KullbackLeibler (KL) divergence 
	\end{itemize}
\end{frame}

\begin{frame}{Future plans: Memorization, Model creation and Inference}
	\begin{itemize}
		\item \textbf{Memorization}: Searching for more studies about different approaches in saving and retrieving predicting models.
		\item \textbf{Model generation}: Searching more studies about existing biological and computational approaches of creating new models from large abnormalities (Model creation, one is introduced in \cite{regazzoni-2020-multi-sensorial-generative-and-descriptive-self-awareness-models-for-autonomous-systems}). 
		\item \textbf{Decision making}: Searching for more studies about different approaches of evolving an observation to a decision in a SA agent.
	\end{itemize}
\end{frame}

\begin{frame}{Collective SA(CA) applications}
	Reviewed Papers:
	\begin{itemize}
		\item Agent collision avoidance  \fullcite{selvaggio-2017-towards-a-self-collision-aware-teleoperation-framework-for-compound-robots}
		\item Traffic jam avoidance \fullcite{qing-2107-real-time-road-traffic-awareness-model-based-on-optimal-multi-channel-self-organized-time-division-multiple-access-algorithm}
		\item Collective incident locating \fullcite{kosak-2019-multipotent-systems-combining-planning-self-organization-and-reconfiguration-in-modular-robot-ensembles}
	\end{itemize}
\end{frame}

\begin{frame}{Semantic CA - Discreeting the continuous state space}
	For semantic segmentation of states (Building the letters), an unsupervised clustering method is needed (Reviewed papers)
	\begin{itemize}
		\item Growing Neural Gas (GNG) online unsupervised clustering  \fullcite{fiser-2013-growing-neural-gas-efficiently}
		\item Self-organizing maps (SOM) \fullcite{kohonen-2001-self-organizing-maps}
	\end{itemize}
	Future plans
	\begin{itemize}
		\item Searching for new improvements in GNGs for online clustering 
	\end{itemize}
\end{frame}

\begin{frame}{CA - Prediction and anomaly detection}
	Discrete state prediction and anomaly detection
	\begin{itemize}
		\item \fullcite{baydoun-2020-prediction-of-multi-target-dynamics-using-discrete-descriptors-an-interactive-approach}
		
		\item \fullcite{kanapram-2019-self-awareness-in-intelligent-vehicles-experience-based-abnormality-detection}
		
		\item \fullcite{kanapram-2020-collective-awareness-for-abnormality-detection-in-connected-autonomous-vehicles}
	\end{itemize}
\end{frame}

\begin{frame}{CA - Future Plans}
	
	\begin{itemize}
		\item More papers from different research groups are planned to be read in future. Example:  \fullcite{dutt-2016-self-awareness-in-cyber-physical-systems}
		\fullcite{esterle-2020-i-think-therefore-you-are-models-for-interaction-in-collectives-of-self-aware-cyber-physical-systems}
		\fullcite{kanapram-2019-dynamic-bayesian-approach-for-decision-making-in-ego-things}
	\end{itemize}
\end{frame}

\begin{frame}{Future plan}
	Reading about semantic implementation of semantic awareness in the following fields
	\begin{itemize}
		\item Reading previous studies which extend CA to heterogeneous robots 
		\item Reading previous studies in modeling other agents' state transition matrix by dispatching semantic discreet composition of E/P states
	\end{itemize}
\end{frame}

\begin{frame}[allowframebreaks]{References}
	\printbibliography
\end{frame}
\end{document}
