\documentclass{article}
\usepackage{comment}
\usepackage[english]{babel}
\usepackage[utf8]{inputenc}
\usepackage{fancyhdr}
\usepackage[round]{natbib}
\usepackage{graphicx}
\usepackage{url}
\usepackage{amsmath}
\usepackage{amssymb}
\DeclareMathOperator*{\argmax}{argmax}
\pagenumbering{arabic}

\pagestyle{fancy}
\fancyhf{}
\rhead{Mohammad Rahmani}
\lhead{Artificial Intelligence}

\newcommand{\ignore}[1]{}
\begin{document}
	\bibliographystyle{plainnat}
	\title{Artificial Intelligence}
	\author{Mohammad Rahmani}
	\date{}
	\maketitle
	Aims at providing a general bio-inspired framework to model SA computationally employing Bayesian
	representations, probabilistic inferences, and machine learning techniques.
	\\
	The proposed framework is capable of comparing
	current experiences with previous ones. 
	\\ 
	It makes it possible the
	learning of descriptive dynamic models that include semantic
	(symbolic) and continuous information, enabling them to
	interpret multisensory observations contextually.
	\\
	Accordingly,
	the proposed SA model represents multisensory experiences
	by using a probabilistic structure that jointly describes ob-
	servations in a semantic, temporal, and hierarchical way.
	\\
	The
	learned models are generative at different levels, in the sense
	that errors obtained from such models (differences between
	already learned experiences and new ones) allow an agent to
	add new models incrementally by describing variations (i.e.,
	errors) at different abstraction levels.
	\\
	 The proposed framework
	can integrate several machine learning techniques (e.g., deep
	neural networks and clustering algorithms) to learn contextual relationships among multisensorial data and model their
	dynamics over time by using a multilevel Bayesian approach.
	\\
	Through section III-A, we describe how the different capa-
	bilities can be jointly obtained by using a unique representation
	and inference approach. A formalization of this approach will
	be discussed in Section III-B, while here, we provide a higher-
	level representation of the proposed solution by introducing
	candidate representation and inference mechanisms.
	
	\section{typos}
		\begin{itemize}
			\item threads
			\item bu
		\end{itemize}
	\bibliography{/media/donkarlo/Elements/projs/research/refs}
\end{document}